%%%%%%%%%%%%%%%%%%%%%%%%%%%%%%%%%%%%%%%
% My Resume based on the: Deedy - One Page Two Column Resume
% LaTeX Template
% Version 1.2 (16/9/2014)
%
% Original author:
% Debarghya Das (http://debarghyadas.com)
%
% Original repository:
% https://github.com/deedydas/Deedy-Resume
%
% IMPORTANT: THIS TEMPLATE NEEDS TO BE COMPILED WITH XeLaTeX
%
% This template uses several fonts not included with Windows/Linux by
% default. If you get compilation errors saying a font is missing, find the line
% on which the font is used and either change it to a font included with your
% operating system or comment the line out to use the default font.
% 
%%%%%%%%%%%%%%%%%%%%%%%%%%%%%%%%%%%%%%
% 
% TODO:
% 1. Integrate biber/bibtex for article citation under publications.
% 2. Figure out a smoother way for the document to flow onto the next page.
% 3. Add styling information for a "Projects/Hacks" section.
% 4. Add location/address information
% 5. Merge OpenFont and MacFonts as a single sty with options.
% 
%%%%%%%%%%%%%%%%%%%%%%%%%%%%%%%%%%%%%%
%
% CHANGELOG:
% v1.1:
% 1. Fixed several compilation bugs with \renewcommand
% 2. Got Open-source fonts (Windows/Linux support)
% 3. Added Last Updated
% 4. Move Title styling into .sty
% 5. Commented .sty file.
%
%%%%%%%%%%%%%%%%%%%%%%%%%%%%%%%%%%%%%%%
%
% Known Issues:
% 1. Overflows onto second page if any column's contents are more than the
% vertical limit
% 2. Hacky space on the first bullet point on the second column.
%
%%%%%%%%%%%%%%%%%%%%%%%%%%%%%%%%%%%%%%


\documentclass[]{deedy-resume-openfont}
\usepackage{fancyhdr}
\usepackage{fontawesome}
 
\pagestyle{fancy}
\fancyhf{}
 
\begin{document}

%%%%%%%%%%%%%%%%%%%%%%%%%%%%%%%%%%%%%%
%
%     LAST UPDATED DATE
%
%%%%%%%%%%%%%%%%%%%%%%%%%%%%%%%%%%%%%%
\lastupdated

%%%%%%%%%%%%%%%%%%%%%%%%%%%%%%%%%%%%%%
%
%     TITLE NAME
%
%%%%%%%%%%%%%%%%%%%%%%%%%%%%%%%%%%%%%%
\namesection{}{Srijal Poojari}{\faGlobe \hspace{1pt} \urlstyle{same}\href{http://srijalpoojari.com}{srijalpoojari.com} \\ 
\href{mailto:srijal97@gmail.com}{ \faEnvelope \hspace{1pt} srijal97@gmail.com} | \faMobile \hspace{0.5pt} +91 9967253367 | \faMapMarker \hspace{1pt} Mumbai, India}

%%%%%%%%%%%%%%%%%%%%%%%%%%%%%%%%%%%%%%
%
%     COLUMN ONE
%
%%%%%%%%%%%%%%%%%%%%%%%%%%%%%%%%%%%%%%

\begin{minipage}[t]{0.33\textwidth} 

%%%%%%%%%%%%%%%%%%%%%%%%%%%%%%%%%%%%%%
%     EDUCATION
%%%%%%%%%%%%%%%%%%%%%%%%%%%%%%%%%%%%%%

\section{Education} 

\subsection{Sardar Patel Institute \newline of Technology}
\descript{BE in Electronics}
University of Mumbai \\
\location{May 2019 | Mumbai, India}
\location{CGPA: 8.49 / 10}
\sectionsep

\subsection{MVM Junior College}
\descript{Higher Secondary Certificate}
Maharashtra State Board \\
\location{Feb 2015 | Mumbai, India}
\location{540/650 (83.08\%)}
\sectionsep

\subsection{St. Xavier's High School}
\descript{Secondary School Certificate}
Maharashtra State Board \\
\location{Mar 2013 | Mumbai, India}
\location{501/550 (91.09\%)}
\sectionsep

%%%%%%%%%%%%%%%%%%%%%%%%%%%%%%%%%%%%%%
%     LINKS
%%%%%%%%%%%%%%%%%%%%%%%%%%%%%%%%%%%%%%

%\section{Links} 
%Facebook:// \href{https://facebook/dd}{\bf dd} \\
%Github:// \href{https://github.com/deedydas}{\bf deedydas} \\
%LinkedIn://  \href{https://www.linkedin.com/in/debarghyadas}{\bf debarghyadas} \\
%YouTube://  \href{https://www.youtube.com/user/DeedyDash007}{\bf DeedyDash007} \\
%Twitter://  \href{https://twitter.com/debarghya_das}{\bf @debarghya\_das} \\
%Quora://  \href{https://www.quora.com/Debarghya-Das}{\bf Debarghya-Das}

%%%%%%%%%%%%%%%%%%%%%%%%%%%%%%%%%%%%%%
%     COURSEWORK
%%%%%%%%%%%%%%%%%%%%%%%%%%%%%%%%%%%%%%

%\section{Coursework}
%\subsection{Graduate}
%Advanced Machine Learning \\
%Open Source Software Engineering \\
%Advanced Interactive Graphics \\
%Compilers + Practicum \\
%Cloud Computing \\
%Defending Computer Networks \\
%Machine Learning \\
%\sectionsep

%\subsection{Undergraduate}
%Information Retrieval \\
%Operating Systems \\
%Artificial Intelligence + Practicum \\
%Functional Programming \\
%Computer Graphics + Practicum \\
%{\footnotesize \textit{\textbf{(Research Asst. \& Teaching Asst 2x) }}} \\
%Unix Tools and Scripting \\

%%%%%%%%%%%%%%%%%%%%%%%%%%%%%%%%%%%%%%
%     SKILLS
%%%%%%%%%%%%%%%%%%%%%%%%%%%%%%%%%%%%%%

\section{Skills}
\subsection{Programming}
\location{Proficient:}
Python \textbullet{}   C \textbullet{} C++ \\
\location{Intermediate:}
MATLAB \textbullet{} \LaTeX \\ %\textbullet{} C\# \textbullet{} Lua 
\location{Familiar:}
C\# \textbullet{} Lua \textbullet{} JavaScript
\sectionsep

\subsection{Hardware}
\location{Development Boards and SoCs:}
Arduino AVR, ARM \textbullet{} ATmega \\
ESP32  \textbullet{} ESP8266 \textbullet{} Microchip dsPIC \\
C2000 DSP \textbullet{} Raspberry Pi \textbullet{} MSP430 \\
Nvidia Jetson TX2, Nano \\
Particle Photon \textbullet{} Spartan V FPGA\\
\location{Design and Development:}
PCB Design \textbullet{} Power PCB Layouts \\
SMD Soldering (QFP, QFN) \textbullet{} 3D Printing \\
\sectionsep

\subsection{Software}
OpenCV \textbullet{} ROS  \textbullet{} VRep \\
EAGLE  \textbullet{} Fusion 360 \textbullet{} Unity\\
\sectionsep

\section{Other Interests}

RC Planes \textbullet{} Aviation \textbullet{} Electronics Salvage\\
Swimming \textbullet{} Reading \textbullet{} Video Games \\
% Art, acrylic painting

\sectionsep

%\section{Links}
%
%\begin{huge}
%	\begin{tabular}{c c c c}
%		\faUser	  & \faGithub  & \faLinkedin &  \faYoutube \\
%	\end{tabular}
%\end{huge}

%%%%%%%%%%%%%%%%%%%%%%%%%%%%%%%%%%%%%%
%
%     COLUMN TWO
%
%%%%%%%%%%%%%%%%%%%%%%%%%%%%%%%%%%%%%%

\end{minipage} 
\hfill
\begin{minipage}[t]{0.66\textwidth} 

%%%%%%%%%%%%%%%%%%%%%%%%%%%%%%%%%%%%%%
%     Projects and Competitions
%%%%%%%%%%%%%%%%%%%%%%%%%%%%%%%%%%%%%%

\section{Projects and Competitions}

\runsubsection{Project ATLAS}
\descript{| BE final year project | \href{http://www.drishti.works/}{Drishti Works}, Mumbai }
\location{Jul '18 - Apr '19}
\vspace{\topsep} % Hacky fix for awkward extra vertical space
\begin{tightemize}
\item Build of a tethered multirotor with an AUW of 10Kgs.
\item Designed 140V to 32V@20A step-down converters in a small form-factor for increasing the altitude capabilities of the system. [Publications pending]
\end{tightemize}
\sectionsep

\runsubsection{Project AURUS}
\descript{| \href{http://www.drishti.works/}{Drishti Works}, Mumbai }
\location{4 Jun – 15 Jul '18}
\begin{tightemize}
\item Worked on the development of AURUS, a beach cleaning robot.
\item Involved in a diverse range of work including programming for the various sensors and functionality, developing the IMU system (MPU6050, MPU9250), the power distribution system, wiring, and several custom PCBs for the robot.
\item Created ROS nodes for communication between the controller and Nvidia Jetson and greatly improved system reliability in the software aspect as well.
\end{tightemize}
\sectionsep

\runsubsection{Hololens Experience Center}
\descript{|  \href{https://fractalanalytics.com/}{Fractal Analytics}, Mumbai}
\location{27 Nov - 26 Dec '17}
\begin{tightemize}
	\item Developed applications on the Microsoft HoloLens Mixed Reality(MR) headsets, in a group of 2.
	\item Conventional statistical results of Share of Shelf, Share of Sight and Compliance for a supermarket shelf are presented in the form of holograms.
	\item Learned app development in Unity using C\#, AR and MR concepts.
\end{tightemize}
\sectionsep

\runsubsection{Modular Robots}
\descript{| IIT Bombay }
\location{22 May – 7 Jul '17}
\begin{tightemize}
	\item Internship to create self-reconfigurable robotic modules, in a group of 2.
	\item 4 robotics modules were created which each consisted of an Arduino Nano, 5 servo motors, bluetooth, RF modules for wireless communication and LiPo batteries for power.
	\item Created a Virtual Dtto interface on VRep with bluetooth control, sensors and behaviour identical to the actual robot.
	\item Also involved designing the 3D CAD models of the modules on Fusion 360, Autodesk Inventor and 3D printing the designs.
\end{tightemize}
\sectionsep

\runsubsection{eYantra Robotics Competiton}
\descript{| IIT Bombay }
\location{Nov '16 - Apr '17}
\begin{tightemize}
\item \textbf{1st out of 162 teams} in the national level competition.
\item Planned and implemented the algorithms for motion planning and used OpenCV with Python for Image Processing.
\item Played the role of team leader in a team of 4. 
\end{tightemize}
\sectionsep

\runsubsection{CYKLO}
\descript{| SPIT, Mumbai}
\location{Nov '15 - May '17}
\begin{tightemize}
\item CYKLO is a point-to-point, peer-to-peer cycle sharing service, started in 2015 at Sardar Patel Institute of Technology(SPIT), Mumbai, with me as a core part of the team.
\item Designed, built and programmed several hardware prototypes for the automated cycle locking system, including the locking mechanism, electronic controller and network interface.
\item \textbf{1st Prize} in InterThrone 2017, an IoT focused contest under the Public Transport category with an award of \textbf{INR 300,000.}
\end{tightemize}
\sectionsep


\end{minipage} 

\newpage

%%%%%%%%%%%%%%%%%%%%%%%%%%%%%%%%%%%%%%
%     Other Work
%%%%%%%%%%%%%%%%%%%%%%%%%%%%%%%%%%%%%%

\section{Other Work}

\runsubsection{Room Occupancy Indicating System}
\descript{| BE Sem VI project }
\location{Feb - Apr 2018}
\begin{tightemize}
\item A network of small wireless sensors to detect human presence in a room, across multiple such rooms. Group of 2.
\item Involved a strong focus on wireless networking, low power design, PCB design, 3D modelling and printing.
\item \textbf{3rd Prize} in Innovatron’18, an inter-college project presentation competition held at SPIT, Mumbai.
\end{tightemize}
\sectionsep

\runsubsection{3D Indoor Mapping}
\descript{| BE Sem V project }
\location{Aug - Oct 2017}
\begin{tightemize}
	\item Used the Microsoft Kinect depth sensor and a Raspberry Pi running the Robot Operating System (ROS) to create a wireless 3D mapping setup.  
\end{tightemize}
\sectionsep

\runsubsection{\href{https://www.instructables.com/id/The-Companion-IC/}{The Companion IC}}
\descript{| Arduino Contest,  \href{https://www.instructables.com/}{Instructables} }
\location{Jan 2016}
\begin{tightemize}
	\item \href{https://www.instructables.com/id/The-Companion-IC/}{The Companion IC} is a lab bench tool for quick testing of electronic components with an aesthetic appeal to it.
	\item \textbf{2nd out of 459 entries} from various parts of the world in the Arduino All-The-Things Contest on Instructables.
\end{tightemize}
\sectionsep

\runsubsection{Circuit Troubleshooting Competition}
\descript{| SPIT, Mumbai}
\location{29 Sep - 17 Oct '16}
\begin{tightemize}
	\item \textbf{1st Prize} in the departmental competition in which students were expected to troubleshoot and rectify faulty electronic circuits; first in simulation on TINA-TI software and later on a breadboard.
\end{tightemize}
\sectionsep


%%%%%%%%%%%%%%%%%%%%%%%%%%%%%%%%%%%%%%
%     Teaching
%%%%%%%%%%%%%%%%%%%%%%%%%%%%%%%%%%%%%%

\section{Teaching and Roles of Responsibility} 
\begin{tabular}{rll}
2019	  & IEEE Workshop on Adding WiFi to your projects using ESP8266 and MQTT  & SPIT, Mumbai   \\
2019	  & Teaching Assistantship: Robotic Vision  & SPIT, Mumbai   \\
2018	  & Departmental Circuit Troubleshooting Competition  & SPIT, Mumbai   \\
2018	  & IEEE Workshop on Introduction to Microcontrollers, Sensors and Arduino  & SPIT, Mumbai   \\
2018	  & Workshop on Introduction to Embedded Systems Design  & SPIT, Mumbai   \\
2016	  & Robocon Workshop on PCB making and basics of a robotic system  & SPIT, Mumbai   \\
\end{tabular}
\sectionsep


%%%%%%%%%%%%%%%%%%%%%%%%%%%%%%%%%%%%%%
%     AWARDS
%%%%%%%%%%%%%%%%%%%%%%%%%%%%%%%%%%%%%%

%\section{Awards} 
%\begin{tabular}{rll}
%2014	     & top 52/2500  & KPCB Engineering Fellow\\
%2014	     & 1\textsuperscript{st}/50  & Microsoft Coding Competition, Cornell\\
%2013	     & National  & Jump Trading Challenge Finalist\\
%2013     & 7\textsuperscript{th}/120 & CS 3410 Cache Race Bot Tournament  \\
%2012     & 2\textsuperscript{nd}/150 & CS 3110 Biannual Intra-Class Bot Tournament \\
%2011     & National & Indian National Mathematics Olympiad (INMO) Finalist \\
%\end{tabular}
%\sectionsep

%%%%%%%%%%%%%%%%%%%%%%%%%%%%%%%%%%%%%%
%     PUBLICATIONS
%%%%%%%%%%%%%%%%%%%%%%%%%%%%%%%%%%%%%%

%\section{Publications} 
%\renewcommand\refname{\vskip -1.5em} % Couldn't get this working from the .cls file
%\bibliographystyle{abbrv}
%\bibliography{publications}
%\nocite{*}

%%%%%%%%%%%%%%%%%%%%%%%%%%%%%%%%%%%%%%
%     Training
%%%%%%%%%%%%%%%%%%%%%%%%%%%%%%%%%%%%%%

\section{Training}

\runsubsection{Robotics: Fundamentals}
\descript{| UPennX, edX }
\location{Oct 2018}
\begin{tightemize}
\item Kinematics and Mathematical Foundations for describing robotic arms and mobile robots using MATLAB.
\end{tightemize}
\sectionsep

\runsubsection{Computation Structures}
\descript{| MITx, edX }
\location{Sep '16 - May '17}
\begin{tightemize}
\item Understanding the principles and techniques used in the design of digital and computer systems. \item Designed a 32-bit ‘Beta’ processor, ground up, from basic logic gates on the Jade simulator.
\end{tightemize}
\sectionsep

\runsubsection{MSP-FPGA Hardware and Software Co-design}
\descript{| SPIT, Mumbai}
\location{Sep 2016}
\begin{tightemize}
\item Workshop on interfacing of MSP430 with an FPGA to enable parallel processing of general purpose calculations on MSP and hardware optimized tasks on FPGA for faster throughput.
\end{tightemize}
\sectionsep

\runsubsection{Embedded Systems Design}
\descript{| SPIT, Mumbai}
\location{Jun 2016}
\begin{tightemize}
	\item Workshop on introduction to various technologies in Embedded Systems with	hands-on practice on development boards including Atmel AVR, ARM, Texas Instruments MSP and DSP.
\end{tightemize}
\sectionsep

\runsubsection{RoboCamp(Sr.) by ThinkLABS}
\descript{| IIT Bombay}
\location{Dec 2010}
\begin{tightemize}
	\item Basic STEM learning for school kids for an early start on autonomous robots and their control using graphical block-programming. 
	\item Interfaced sensors like touch, IR and control of DC motor for robot actuation was learnt on the iPitara Robot by ThinkLABS. Participated in TRICKS 2010, IIT Bombay.
\end{tightemize}
\sectionsep


\end{document}  \documentclass[]{article}
