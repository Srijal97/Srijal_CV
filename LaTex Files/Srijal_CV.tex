%%%%%%%%%%%%%%%%%%%%%%%%%%%%%%%%%%%%%%%
% My Resume based on the: Deedy - One Page Two Column Resume
% LaTeX Template
% Version 1.2 (16/9/2014)
%
% Original author:
% Debarghya Das (http://debarghyadas.com)
%
% Original repository:
% https://github.com/deedydas/Deedy-Resume
%
% IMPORTANT: THIS TEMPLATE NEEDS TO BE COMPILED WITH XeLaTeX
%
% This template uses several fonts not included with Windows/Linux by
% default. If you get compilation errors saying a font is missing, find the line
% on which the font is used and either change it to a font included with your
% operating system or comment the line out to use the default font.
% 
%%%%%%%%%%%%%%%%%%%%%%%%%%%%%%%%%%%%%%
% 
% TODO:
% 1. Integrate biber/bibtex for article citation under publications.
% 2. Figure out a smoother way for the document to flow onto the next page.
% 3. Add styling information for a "Projects/Hacks" section.
% 4. Add location/address information
% 5. Merge OpenFont and MacFonts as a single sty with options.
% 
%%%%%%%%%%%%%%%%%%%%%%%%%%%%%%%%%%%%%%
%
% CHANGELOG:
% v1.1:
% 1. Fixed several compilation bugs with \renewcommand
% 2. Got Open-source fonts (Windows/Linux support)
% 3. Added Last Updated
% 4. Move Title styling into .sty
% 5. Commented .sty file.
%
%%%%%%%%%%%%%%%%%%%%%%%%%%%%%%%%%%%%%%%
%
% Known Issues:
% 1. Overflows onto second page if any column's contents are more than the
% vertical limit
% 2. Hacky space on the first bullet point on the second column.
%
%%%%%%%%%%%%%%%%%%%%%%%%%%%%%%%%%%%%%%


\documentclass[]{deedy-resume-openfont}
\usepackage{fancyhdr}
\usepackage{fontawesome}
\usepackage{multirow}
\usepackage{makecell}

\usepackage{array}
\newcolumntype{L}[1]{>{\raggedright\let\newline\\\arraybackslash}m{#1}}
\newcolumntype{C}[1]{>{\centering\let\newline\\\arraybackslash\hspace{0pt}}m{#1}}
\newcolumntype{R}[1]{>{\raggedleft\let\newline\\\arraybackslash\hspace{0pt}}m{#1}}

 
\pagestyle{fancy}
\fancyhf{}
 
\begin{document}

%%%%%%%%%%%%%%%%%%%%%%%%%%%%%%%%%%%%%%
%
%     LAST UPDATED DATE
%
%%%%%%%%%%%%%%%%%%%%%%%%%%%%%%%%%%%%%%%
\lastupdated

%%%%%%%%%%%%%%%%%%%%%%%%%%%%%%%%%%%%%%
%
%     TITLE NAME
%
%%%%%%%%%%%%%%%%%%%%%%%%%%%%%%%%%%%%%%
\namesection{}{Srijal Poojari}{\faGlobe \hspace{1pt} \urlstyle{same}\href{http://srijalpoojari.com}{srijalpoojari.com} \\ 
\href{mailto:srijal97@gmail.com}{ \faEnvelope \hspace{1pt} srijal97@gmail.com} | \faMobile \hspace{0.5pt} +91 9967253367 | \faMapMarker \hspace{1pt} Mumbai, India}

%%%%%%%%%%%%%%%%%%%%%%%%%%%%%%%%%%%%%%
%
%     COLUMN ONE
%
%%%%%%%%%%%%%%%%%%%%%%%%%%%%%%%%%%%%%%

\begin{minipage}[t]{0.33\textwidth} 

%%%%%%%%%%%%%%%%%%%%%%%%%%%%%%%%%%%%%%
%     EDUCATION
%%%%%%%%%%%%%%%%%%%%%%%%%%%%%%%%%%%%%%

\section{Education} 

\subsection{Sardar Patel Institute \newline of Technology}
\descript{BE in Electronics}
University of Mumbai \\
\location{May 2019 | Mumbai, India}
\location{CGPA: 8.49 / 10}
\sectionsep

\subsection{MVM Junior College}
\descript{Higher Secondary Certificate}
Maharashtra State Board \\
\location{Feb 2015 | Mumbai, India}
\location{540/650 (83.08\%)}
\sectionsep

\subsection{St. Xavier's High School}
\descript{Secondary School Certificate}
Maharashtra State Board \\
\location{Mar 2013 | Mumbai, India}
\location{501/550 (91.09\%)}
\sectionsep

%%%%%%%%%%%%%%%%%%%%%%%%%%%%%%%%%%%%%%
%     LINKS
%%%%%%%%%%%%%%%%%%%%%%%%%%%%%%%%%%%%%%

%\section{Links} 
%Facebook:// \href{https://facebook/dd}{\bf dd} \\
%Github:// \href{https://github.com/deedydas}{\bf deedydas} \\
%LinkedIn://  \href{https://www.linkedin.com/in/debarghyadas}{\bf debarghyadas} \\
%YouTube://  \href{https://www.youtube.com/user/DeedyDash007}{\bf DeedyDash007} \\
%Twitter://  \href{https://twitter.com/debarghya_das}{\bf @debarghya\_das} \\
%Quora://  \href{https://www.quora.com/Debarghya-Das}{\bf Debarghya-Das}

%%%%%%%%%%%%%%%%%%%%%%%%%%%%%%%%%%%%%%
%     COURSEWORK
%%%%%%%%%%%%%%%%%%%%%%%%%%%%%%%%%%%%%%

%\section{Coursework}
%\subsection{Graduate}
%Advanced Machine Learning \\
%Open Source Software Engineering \\
%Advanced Interactive Graphics \\
%Compilers + Practicum \\
%Cloud Computing \\
%Defending Computer Networks \\
%Machine Learning \\
%\sectionsep

%\subsection{Undergraduate}
%Information Retrieval \\
%Operating Systems \\
%Artificial Intelligence + Practicum \\
%Functional Programming \\
%Computer Graphics + Practicum \\
%{\footnotesize \textit{\textbf{(Research Asst. \& Teaching Asst 2x) }}} \\
%Unix Tools and Scripting \\

%%%%%%%%%%%%%%%%%%%%%%%%%%%%%%%%%%%%%%
%     SKILLS
%%%%%%%%%%%%%%%%%%%%%%%%%%%%%%%%%%%%%%

\section{Skills}
\subsection{Programming}
\location{Proficient:}
Python \textbullet{}   C \textbullet{} C++ \\
\location{Intermediate:}
MATLAB \textbullet{} \LaTeX \\ %\textbullet{} C\# \textbullet{} Lua 
\location{Familiar:}
C\# \textbullet{} Lua \textbullet{} JavaScript
\sectionsep

\subsection{Hardware}
\location{Development Boards and SoCs:}
Arduino AVR, ARM \textbullet{} ATmega \\
ESP32  \textbullet{} ESP8266 \textbullet{} Microchip dsPIC \\
C2000 DSP \textbullet{} Raspberry Pi \textbullet{} MSP430 \\
Nvidia Jetson TX2, Nano \\
Particle Photon \textbullet{} Spartan V FPGA\\
\location{Design and Development:}
PCB Design \textbullet{} SMD Soldering (QFP, QFN) \\
Power PCB Layouts \textbullet{} 3D Printing \\
\sectionsep

\subsection{Software}
OpenCV \textbullet{} ROS  \textbullet{} VRep \\
EAGLE  \textbullet{} Fusion 360 \textbullet{} Unity\\
\sectionsep

\section{Other Interests}

RC Planes \textbullet{} Aviation \textbullet{} Electronics Salvage\\
Swimming \textbullet{} Reading \textbullet{} Video Games \\
% Art, acrylic painting

\sectionsep

%\section{Links}
%
%\begin{huge}
%	\begin{tabular}{c c c c}
%		\faUser	  & \faGithub  & \faLinkedin &  \faYoutube \\
%	\end{tabular}
%\end{huge}

%%%%%%%%%%%%%%%%%%%%%%%%%%%%%%%%%%%%%%
%
%     COLUMN TWO
%
%%%%%%%%%%%%%%%%%%%%%%%%%%%%%%%%%%%%%%

\end{minipage} 
\hfill
\begin{minipage}[t]{0.66\textwidth} 

%%%%%%%%%%%%%%%%%%%%%%%%%%%%%%%%%%%%%%
%     Achievements
%%%%%%%%%%%%%%%%%%%%%%%%%%%%%%%%%%%%%%

\section{Achievements}
\vspace{\topsep} % Hacky fix for awkward extra vertical space

\begin{tightemize}
	\item \textbf{2nd Prize in Technical Paper Competition} at SPIT, Mumbai (2019)
	
	\item \textbf{3rd Prize in Innovatron’18}, an inter-college project competition held at SPIT, Mumbai, for the project ``Room Occupancy Indicating System" (2018)
	
	\item \textbf{1st out of 162 teams nationwide} in e-Yantra Robotics Competiton (2016-17)
	
	\item \textbf{1st Prize in InterThrone 2017} an IoT focused contest. Award of INR 300,000 given for the automated cycle locking prototypes developed for CYKLO (2017)
	
	\item \textbf{1st Prize in Circuit Troubleshooting Competition} at SPIT, Mumbai (2016)
	
	
	\item \textbf{2nd out of 459 entries worldwide} in the Arduino All-The-Things Contest on \\ Instructables, for the project of ``The Companion IC" \href{https://www.instructables.com/id/The-Companion-IC/}{ \faExternalLink} (2016)
	
	\item \textbf{1st Prize in CodeChamps}, a programming competition across all departments of SPIT, Mumbai. Language of choice used was C++ (2015)
	
\end{tightemize}
\sectionsep

%%%%%%%%%%%%%%%%%%%%%%%%%%%%%%%%%%%%%%
%     Teaching
%%%%%%%%%%%%%%%%%%%%%%%%%%%%%%%%%%%%%%

\section{Teaching} 
\begin{tabular}{r L{8 cm} l}
	
	% used \makecell to use newline inside of a cell and \phantom{.} for an invisible character. Without this, the text is vertically center aligned.
	
	2019	  & Workshop on Signal and Image Processing on DSPs  & 	\makecell[l]{MPSTME, Mumbai}   \\
	
	2019	  & Teaching Assistantship: Product Design  & \makecell[l]{SPIT, Mumbai}   \\
	
	\makecell{2019 \\ \phantom{.}}  & IEEE Workshop on adding WiFi to your projects using ESP8266 and MQTT  & \makecell[l]{SPIT, Mumbai \\ \phantom{.}}   \\
	
	2019	  & Teaching Assistantship: Robotic Vision  & \makecell[l]{SPIT, Mumbai}   \\
	
	\makecell{2018 \\ \phantom{.}}	  & IEEE Workshop on Introduction to Microcontrollers, Sensors and Arduino  & \makecell[l]{SPIT, Mumbai \\ \phantom{.}}   \\
	
	\makecell{2018 \\ \phantom{.}}	  & Workshop on Introduction to Embedded Systems Design  & \makecell[l]{SPIT, Mumbai \\ \phantom{.}}   \\
	
	\makecell{2016 \\ \phantom{.}}	  & Robocon Workshop on PCB making and basics of a robotic system  & \makecell[l]{SPIT, Mumbai \\ \phantom{.}}   \\
\end{tabular}
\sectionsep



%%%%%%%%%%%%%%%%%%%%%%%%%%%%%%%%%%%%%%
%     Work Experience
%%%%%%%%%%%%%%%%%%%%%%%%%%%%%%%%%%%%%%

\section{Work Experience}

\runsubsection{SP Product Development Cell}
\descript{| Research Associate }
\location{July '19 onwards | SPIT, Mumbai}

Work on industry consultancy projects under Dr. R. R. Sawant and Dr. Y. S. Rao. Involves development of embedded systems of different microcontroller families like dsPIC, C2000 and ATmega that usually deal with high power electronics.

\sectionsep

\runsubsection{\href{http://www.drishti.works/}{Drishti Works}}
\descript{| Intern - Robotics Engineer }
\location{4 Jun – 15 Jul '18 | Mumbai}

Developed the sensing, power distribution and IMU system for AURUS, a beach cleaning robot. Optimized the computing stack and sensing + actuation stack communication in ROS.

\sectionsep

\runsubsection{\href{https://fractalanalytics.com/}{Fractal Analytics}}
\descript{| Project Intern}
\location{27 Nov - 26 Dec '17 | Mumbai}

Developed applications using Unity (C\#) on the Microsoft HoloLens Mixed Reality(MR) headsets for displaying conventional statistical results in the form of holograms.

\sectionsep

\runsubsection{IIT Bombay}
\descript{| Summer Intern - Modular Robots}
\location{22 May - 7 Jul '17 | Mumbai}

Created self-reconfigurable robot modules inspired by the Dtto Robot. Developed virtual simulations of the same modules on VRep with bluetooth control.

\sectionsep


\end{minipage} 

\newpage


%%%%%%%%%%%%%%%%%%%%%%%%%%%%%%%%%%%%%%
%     Projects and Competitions
%%%%%%%%%%%%%%%%%%%%%%%%%%%%%%%%%%%%%%

\section{Relevant Projects}

\runsubsection{Project ATLAS}
\descript{| BE final year project}
\location{Jul '18 - Apr '19 | \href{http://www.drishti.works/}{Drishti Works}, Mumbai }
\begin{tightemize}
	\item Build of a tethered multirotor with an AUW of 10Kgs.
	\item Designed fully custom 140V to 32V@20A step-down switching converters, at a high frequency of 855kHz, for a small form-factor and low weight. [Final Report \href{http://bit.ly/ATLAS-Report}{ \faExternalLink}]
\end{tightemize}
\sectionsep

\runsubsection{TinyBot}
\descript{| Personal Project }
\location{Mar - Nov '19}
\begin{tightemize}
	\item Created a very small wireless remote-controlled car of the order of 10*10*10mm as a challenge to a reddit post.
	\item Custom designed a PCB for the power regulators and motor drivers, to be used along with the ESP8285. I believe it to be the smallest hobby-level RC car. 
	% I am limited by the technology of my time...or perhaps money.
\end{tightemize}
\sectionsep

\runsubsection{Room Occupancy Indicating System}
\descript{| BE Sem VI project }
\location{Feb - Apr '18 | SPIT, Mumbai}
\begin{tightemize}
	\item A network of small wireless sensors to detect human presence in a room, across multiple such rooms.
	\item Involved a strong focus on wireless networking, low power design, PCB design, 3D modelling and printing.
\end{tightemize}
\sectionsep

\runsubsection{3D Indoor Mapping}
\descript{| BE Sem V project }
\location{Aug - Oct '17 | SPIT, Mumbai}
\begin{tightemize}
	\item Used the Microsoft Kinect depth sensor and a Raspberry Pi running the Robot Operating System (ROS) to create a wireless 3D mapping setup.  
\end{tightemize}
\sectionsep

\runsubsection{eYantra Robotics Competiton}
\descript{\phantom{.}}
\location{Nov '16 - Apr '17 | IIT Bombay}
\begin{tightemize}
	\item Planned and implemented the algorithms for motion planning and used OpenCV with Python for Image Processing.
	\item Played the role of team leader in a team of 4. [Demo Video \href{http://bit.ly/eyrc-demo}{\faYoutubePlay}]
\end{tightemize}
\sectionsep

\runsubsection{CYKLO}
\descript{| Startup}
\location{Nov '15 - May '17 | SPIT, Mumbai}
\begin{tightemize}
	\item CYKLO is a point-to-point cycle sharing service, started in 2015 at SPIT, Mumbai, with me as a core part of the team.
	\item Designed, built and programmed several hardware prototypes for the automated cycle locking system, including the locking mechanism, electronic controller and network interface.
\end{tightemize}
\sectionsep

%%%%%%%%%%%%%%%%%%%%%%%%%%%%%%%%%%%%%%
%     Training
%%%%%%%%%%%%%%%%%%%%%%%%%%%%%%%%%%%%%%

\section{Training}

\runsubsection{Robotics: Fundamentals}
\descript{| UPennX, edX - Certified \href{https://courses.edx.org/certificates/c152ee504c954a208a262c2fafe5aca3}{ \faExternalLink}}
\location{Oct 2018}
Kinematics and Mathematical foundations for describing robotic arms and mobile robots using MATLAB.

\sectionsep

\runsubsection{Computation Structures}
\descript{| MITx, edX - Uncertified }
\location{Sep '16 - May '17}
 Designed a 32-bit ‘Beta’ processor, ground up, from basic logic gates on the Jade simulator.

\sectionsep

\runsubsection{MSP-FPGA Hardware and Software Co-design}
\descript{| SPIT, Mumbai}
\location{Sep 2016}
 Workshop on interfacing of MSP430 with an FPGA to enable parallel processing of general purpose calculations on MSP and hardware optimized tasks on FPGA for faster throughput.
\sectionsep

\runsubsection{Embedded Systems Design}
\descript{| SPIT, Mumbai}
\location{Jun 2016}
 Workshop on introduction to various technologies in Embedded Systems with	hands-on practice on development boards including Atmel AVR, ARM, Texas Instruments MSP and DSP.

\sectionsep

\runsubsection{RoboCamp(Sr.) by ThinkLABS}
\descript{| IIT Bombay}
\location{Dec 2010}
Basic STEM learning for school kids on autonomous robots by interfacing sensors like touch and IR, and control of DC motor for actuation on the iPitara Robot by ThinkLABS. Participated in TRICKS 2010, IIT Bombay.
\sectionsep

%%%%%%%%%%%%%%%%%%%%%%%%%%%%%%%%%%%%%%
%     AWARDS
%%%%%%%%%%%%%%%%%%%%%%%%%%%%%%%%%%%%%%

%\section{Awards} 
%\begin{tabular}{rll}
%2014	     & top 52/2500  & KPCB Engineering Fellow\\
%2014	     & 1\textsuperscript{st}/50  & Microsoft Coding Competition, Cornell\\
%2013	     & National  & Jump Trading Challenge Finalist\\
%2013     & 7\textsuperscript{th}/120 & CS 3410 Cache Race Bot Tournament  \\
%2012     & 2\textsuperscript{nd}/150 & CS 3110 Biannual Intra-Class Bot Tournament \\
%2011     & National & Indian National Mathematics Olympiad (INMO) Finalist \\
%\end{tabular}
%\sectionsep

%%%%%%%%%%%%%%%%%%%%%%%%%%%%%%%%%%%%%%
%     PUBLICATIONS
%%%%%%%%%%%%%%%%%%%%%%%%%%%%%%%%%%%%%%

%\section{Publications} 
%\renewcommand\refname{\vskip -1.5em} % Couldn't get this working from the .cls file
%\bibliographystyle{abbrv}
%\bibliography{publications}
%\nocite{*}




\end{document}  \documentclass[]{article}
